\section{Introduction}
% problem, objective, aims, !self-contained document!, challenges, solutions, how effective for the solutions
\subsection{Addressed Problem}
Population protocol is a theoretical model for distributed computation \cite{AspnesR2007, MCS11}.
The model contains a collection of indistinguishable agents. The network constructor \cite{MS16a} and the
terminating grid network constructor \cite{Mi17} are some models extending population protocol but with a different aim to construct network in different topologies.
There was a previous work called \textit{netcs} to simulate models of network constructor but there is no attempts
to simulate protocols for terminating grid network constructor and produce integrated simulator containing all these three models.
Hence, this project will attempt to simulate some models of terminating grid network constructor, create an integrated simulator and
use the simulator to observe some protocols of these three models mentioned above.

\subsection{Aims and Objectives}
\par\noindent
The project aimed to study general population protocols \cite{AspnesR2007} and
its two derived model,
network constructor \cite{MS16a} and terminating grid network constructor. \cite{Mi17}
It also attempted to experimentally simulate and visualise these protocols
via building an extensible simulator and visualizer.

\subsection{The challenges in the project}
\subsubsection{Heterogeneous for different types of Models}

\par\noindent
The theoretical models involved in three main different models initially originated in
population protocols. These three models share inherently common points but there are also some
conceptual differences in between them. For instance, the network constructor \cite{MS16a} and terminating
grid network constructor \cite{Mi17}
involve state of connections in between two nodes while the original population protocol does not.
The node of terminating grid network constructor has its complexity structurally compared with
the other two types of model.

\subsubsection{Heterogeneous for different types of Protocols}

\par\noindent
The protocols discussed in the related papers \cite{AspnesR2007, MS16a, Mi17} involve
many different protocols. The protocols are totally different on many characteristics,
such as their different computational ability, different ending way in either convergence or termination and different
computation target. These differences from
protocol to protocol may lead the simulator and visualizer hard to develop and test.

\subsection{Produced solution: Brief introduction to the programme}

\par\noindent
The final programme contains an UI with an fix-sized area to illustrate the interaction process of
a particular protocol and shows the states of elements\footnote{\noindent "Elements" refers nodes in general population protocol,
but also includes edge if the protocol involves edge states.} of the population.
In addition, it contains an information panel presenting some related
information with regard to the population itself, including:
\begin{itemize}
  \item Number of nodes
  \item Number of nodes distinguished in different status
  \item Number of selections for pairs of nodes\footnote{may also include pair of ports for terminating grid network constructor} that scheduler had took
  \item Number of effective interactions the population executed
\end{itemize}

\par\noindent
Additionally, it provides a set of parameters' settings regarding the initial configuration of a protocol to be simulated, which includes:
\begin{itemize}
  \item The number of nodes included in the simulation
  \item The initial state for each node\footnote{\noindent The state of edge for network constructors (i.e. network constructor and terminating grid network constructor) should be always "0" (i.e. inactivated) at initial, so it is omitted here.}
  \item The protocol type (and also different sets of transition rules correspondingly for the protocol)
  \item Option on whether to use fast-forward simulation method for initially $n$-times selection, and the value of $n$ if the option is enabled.
  A fast-forward simulation executed in the model but does not present that process in the viewer so it normally faster than the case that does not enable this option.
\end{itemize}

\par\noindent
The simulator is able to simulate these three different types of models mentioned before and allows user to define their own protocols using simplified
kotlin language with some specific conventions.

\par\noindent
The simulator finally exams a series of defined protocols and find that the dancing protocol will hard to converge under the case that number of leaders are slightly more than the followers
in its initial configuration.

\subsection{Testing of the project}

\par\noindent
In general, the simulator contains two main partition, the viewer (user interface) and the model.

\paragraph{Viewer, or User Interface} The UI functions of simulator is verified through a large number of different population simulations. This ensures the UI functions work as they expected in design stage.
These experiments on theoretical model may also be asserted the correctness of model through the output configuration of these simulations.

\paragraph{Model} The model partition of the simulator developed through Testing driven development method. According to the specification, the unit tests written in JUnit \cite{JUnit} had been written before the written of any
model code . The model code has to pass all testing suits after their implementation.

\subsection{Evaluation of the project}
In general, the project is successful because the delivered version had finished most of the functionalities. It is still have an issue when
attempting to simulate protocols in terminating grid network constructor causing from the current implementation would ignore the axisymmetric cases for some situation and sometimes may lead to
over-rejection. Also, the Domain Specific Language is not been defined and its parser finally is given up in the new design. Though there are some flaws and improvement points existing, the project generally produces a working simulator as it
proposed to be within the limited time of implementation. The experiments on dancing protocol also demonstrate that the simulator can assist the studies or researches
related to population protocol.
